\section{User Study}

We evaluated the system by user studies. The purpose of the evaluation is to test the hypothesis that resume-job matching approach could return better results than keywords searching approach.

Every student will be given a resume, and will spend approximately 10 minutes to familiar with it.
Then they will be assigned a task to search a job with the key-word searching system and the resume-job matching system. Finally, they will fill out the post-evaluation regarding their experience with the two system.

\subsection{Procedure of User Study}

Before testing, users were introduced to the JobFinder. The investor showed the users how to use the three methods in the system to search the jobs.

At the start of this study, a subject was assigned a resume and a job searching task.  The subject was asked to search a specific kind of job, like web developer or software engineer. He/she can use 10 minutes to read the resume to familiar with it.  The subject will be asked to use the keyword job searching system at first,  he need find 5 jobs that he think they are relative appropriate to the given resume. Then subject was asked to perform the same searching task with the resume-job matching system. The jobs the subject find and the time the subject used with the resume-job matching system was recorded as well. At the end of this study, the subject was asked to take a survey that asks about the personal judgment about the results accuracy of the three methods. 


\subsection{Results}
Five users participated user study. The users were graduate students of Computer Science department of Texas A\&M University, all of them can understand the meanings of technical terms of job description.



