\chapter{EVALUATION}

To evaluate the system, some measures will be used. We also proposed two evaluation method: Pre-collected Data and User's direct experience.

\subsection{Basic measures of the system}

In traditional information retrieval system, some measures are widely used~\cite{manning2008introduction}. These measures include:

\begin{enumerate}
    \item Precision ($P$) is the fraction of retrieved documents that are relevant .
       $$  Precision =  \frac{ \#(releveant~items~ retrieved)}{ \#(retrieved~items)}$$
    \item Recall ($R$) is the fraction of relevant documents that are retrieved.
       $$  Recall =  \frac{ \#(releveant~items~ retrieved)}{ \#(releveant~items)}$$
    \item $F measure$ ($F_1 score$) trades off precision versus recall.
       $$ F_1 = 2 \cdot \frac{ Precision \cdot Recall}{ Precision + Recall } $$
    \item Since the results are ranked, $ Normalized~Discounted~Cumulative~Gain ( NDCG )$ will be an important measure to evaluate the ranked retrieval results. For a set of queries $Q$, let $R(j,d)$ be the relevance score assessors gave to document $d$ for query $j$.
       $$ NDCG(Q,k) = \frac {1}{|Q|} \sum_{j=1}^{|Q|}{Z_{kj}} \sum_{m=1}^{k} \frac{2^{R(j,m)} - 1}{ \log_2(1+m)} $$
where $Z_{kj}$ is a normalization factor calculated to make it so that a perfect ranking's NDCG at $k$ for query $j$ is 1. For queries for which $k' < k$ documents are retrieved, the last summation is done up to $k'$.

\end{enumerate}

\subsection{Pre-collected Data}

Some resumes will be pre-collected, and some their matched jobs will be found manually. These jobs will be put into the job database with some other unrelated jobs.  When searching the jobs with the resume, we can get precision, recall and F measure of the system. With ranking positions of the searching results we could calculate the NDCG measure.

\subsection{User Study}
Users will be asked to use both the system and current job finding web site. We can compare some factors to evaluate the system, such as:
\begin{enumerate}
    \item The time consumed to find satisfying jobs.
    \item The satisfaction of search results.
    \item The user subjective experience with the both systems.
\end{enumerate}
