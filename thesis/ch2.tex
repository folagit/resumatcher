\chapter{BACKGROUND}


Some scholars found that current Boolean search and filtering techniques cannot satisfy the complexity of candidate-job matching requirement.~\cite{malinowski2006matching} They hope the system could understand the job requirement, determine which requirements are mandatory and which are optional but preferable. So they moved to use recommender systems technique to address the problem of information overflow.

\section{Job Recommend System}

Job searching is not a new topic in information retrieval, which has been the focus of some commercial job finding web sites and research papers. Usually scholars called them Job Recommend System (JRS), because most of them used technologies from recommender systems. Wei et al. classified Recommend Systems into four categories~\cite{wei2007survey}: Collaborative Filtering, Content-based filtering, Knowledge-based and Hybrid approaches. Some of these techniques had been applied into JRS; Zheng et al. ~\cite{siting2012job} and AlOtaibi et al.~\cite{al2012survey} summarized the categories of existing online recruiting platforms and listed the advantages and disadvantages of technical approaches in different JRSs. The categories include:

\begin{enumerate}
    \item Content-based Recommendation (CBR) The principle of a content-based recommendation is to suggest items that have similar content information to the corresponding users, like Prospect \cite{singh2010prospect}.

    \item Collaborative Filtering Recommendation (CFR) Collaborative filtering recommendation, which finds  similar  users  who have  the same taste with the target user and recommends items based on what the similar users, like CASPER~\cite{rafter2000personalised}.

    \item Knowledge-based Recommendation (KBR) In the knowledge-based recommendation, rules and patterns obtained from the functional knowledge of how a specific item  meets the requirement of a particular user, are used for recommending items, like  Proactive~\cite{lee2007fighting}.

\end{enumerate}

There are two main challenges in Content-Based Recommendation Systems: One is how to extract the information from the job descriptions and job seekers' resumes, the other is how to calculate similarity of them.


\section{Information Extraction}
One important the problem of this system is how to build the models from Job Description and Resume. The first step of model generating is information extraction. Both resumes and job descriptions are written in natural language, so we need to extract information from such un-structured or semi-structured data source, and transfer them to some designed structure.

\subsection{Machine learning and Rule-based IE technologies}

Chiticariu et al. \cite{chiticariu2013rule}summarized the pros and cons of machine learning (ML) and rule-based IE technologies in Table

\begin{table}[ht]
\caption{Pros and Cons of ML and Rule-Based IE technologies } % title of Table
\centering % used for centering table
\begin{tabular}{ | c |  p{6cm} | p{6cm} | } % centered columns (4 columns)
\hline  %inserts double horizontal lines
 & Pros  & Cons  \\ [0.5ex] % inserts table
%heading
\hline % inserts single horizontal line
Rule-based &
    \begin{packed_itemize}
      \item Declarative
      \item Easy to comprehend
      \item Easy to maintain
      \item Easy to incorporate domain knowledge
      \item Easy to trace and fix the cause of errors
    \end{packed_itemize}
    &
     \begin{packed_itemize}
      \item Heuristic
      \item Requires tedious manual labor
    \end{packed_itemize} \\

\hline
ML-based   &
    \begin{packed_itemize}
      \item Trainable
      \item Adaptable
      \item Reduces manual effort
    \end{packed_itemize}
    &
     \begin{packed_itemize}
      \item Requires labeled data
      \item Requires retraining for domain adaptation
      \item Requires ML expertise to use or maintain
      \item Opaque
    \end{packed_itemize} \\

\hline %inserts single line
\end{tabular}
\label{tab:mlrb} % is used to refer this table in the text
\end{table}

Yu et al.~\cite{yu2005resume} used a cascaded information extraction (IE) framework to get the detailed information from the job seeker��s resume. In the first stage, the Hidden Markov Modeling (HMM) model is used to segment the resume into consecutive blocks. Based on the result, a SVM model is used to obtain the detailed information in the certain block, the information include: name, address, education etc.

 Celik Duygua and Elci Atilla proposed a Ontology-based R��sum�� Parser (ORP) ~\cite{ccelik2013ontology}, which uses ontology to assistant the information extraction process. The system process the resume in following steps: convert the resume files into plain text, separate the text into  some segments, use Ontology Knowledge Base to find the concepts in the sentences, normalize all the terms, at last the system will classify the sentences to get the wanted terms.

\section{Matching Algorithms}

Lu et al~\cite{lu2013recommender}. used latent Semantic Analysis(LSA) to calculate similarities between jobs and candidates, but they only selected two factors ``interest'' and ``education''  to compare candidates. Xing et al. ~\cite{yi2007matching} used Structured Relevance Models (SRM) to  match resumes and jobs.

The Ontology technics had also been used in some JRSs, which had been well studied by  Shvaiko and Euzenat in \cite{shvaiko2013ontology}.  Proactive~\cite{lee2007fighting} used two kinds of ontology, job category and the company information. The system used ontology checker to classify the job information, stored the domain knowledge and calculated the weight value in recommendations.

Kumaran et al~\cite{kumaran2013towards} also used ontology to calculate the similarity between job criteria and candidates's resume in their system~\cite{kumaran2013towards}. The similarity equation they used are:
$$ M\left ( i_1, i_2 \right ) = \frac{\sum_{k=1}^{n} Sim\left (p_{k}^{i1},  p_{k}^{i2} \right ) * W_{k}^{i2}}{\sum_{k=1}^{n} W_{k}^{i2}}  $$
The similarity function $Sim(p_1, p_2)$ is defined as follows:
$$ Sim(p1, p2) = \begin{Bmatrix}
1, & if~similarity~of~p1~and~p2 \geqslant t\\
0, & otherwise
\end{Bmatrix} $$

Fazel \cite{fazel2009semantic} used a hybrid approach to matching job seekers and job postings, which takes advantage of the benefits of both logic-based and ontology-based matching. In his paper the description logics (DL) is used to represent the candidate and job opening, and the ontology is used to organize the skills in a skill taxonomy. He gave an equation to calculate the matching degree:
$$ sim\left(P ,j \right) = \sum x_{ij} \times u(ds_i) $$

where, $x_{ji}$ is the Boolean variable indicating whether desire i is satisfied by applicant $A_{j}$ in the set of all qualified applications.

Liu and Dew~\cite{liu2004using} used RDF to represent and store the expertise of experts , and a RDF-based Expertise Matcher could retrieves the experts whose expertise include the required concept.
