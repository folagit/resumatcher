\chapter{Ontology Similarity}


We notice that simple keyword of skill name matching is far from enough, because job description and resume are both written in human language, even the same concepts, they could be written in different ways. For example, Table~\ref{tab:resume_jd}  is part of the resume of a job seeker, and part of a job description:

\begin{table}[ht]
\caption{Resume and Job Description} % title of Table
\centering % used for centering table
\begin{tabular}{ | p{8cm} | p{7cm} | }
 \hline
   \textbf{Part of Resume}                 &   \textbf{Part of Job Description}   \\ \hline

    B.S. degree in computer science \newline
    5+ years Java \newline
    2+ year   C++  \newline
    Some experience in Oracle database \newline
    Other experience like: \newline
    Hibernate, JBOSS, JUnit, Tomcat etc.
  &
  BS degree above   \newline
  4+ years Java  \newline
  Some experience of Python   \newline
    Mysql, MS-SQL   \newline
    Java web application Server   \newline
    OOA/OOD   \\
 \hline
\end{tabular}
\label{tab:resume_jd} % is used to refer this table in the text
\end{table}

If just looking at the text, we can find the resume has few common words with the job description.  But from the view of an experienced engineer, the candidate is pretty matching the job. Because relational databases Oracle and Mysql are very similar, OOA/OOD is the same meaning of many years of Java and C++ experience, and Tomcat and JBOSS are two Java web applications servers.  If we use key word matching, the system won't give a good matching result in this very common situation. So when design the new ontology matching algorithm, we have such considerations:

\begin{enumerate}
    \item How to normalize the same concept with different name or spelling
    \item If one concept in the job description but there is no the same one in the resume, how to calculate the similarity between its related concepts.
    \item If one concept in the job description has multiple similar concepts in the resume, how to summarize total similarity of them.
    \item When calculating the similarity between resume and job description, how to give weights to each concepts.
\end{enumerate}

\section{Ontology Construction}

Semantic web have been a hot research topic in these years, thousands of domain ontologies had created~\cite{ding2004swoogle}. A paradigmatic example is WordNet~\cite{fellbaum1998wordnet}, is a general purpose thesaurus, that contains more than 100,00 general English concepts.

\section{statistical-based Ontology Similarity Measure }
In this thesis, I proposed a new statistical-based ontology similarity measure. In most of job descriptions, they will list many skills the positions required. From observation, we could find that related skills always exist in a job description simultaneously, e.g. HTML and CSS are always required together. Furthermore, the distance between two concepts is also a good measure of the relevance of them. We could see this phenomenon in table~\ref{tab:skillinsent}, which include some skills requirement sentences from some job desertions :

\begin{table}[ht]
\caption{Some sentences Job Description} % title of Table
\centering % used for centering table
\begin{tabular}{ | p{15cm}  | }
 \hline
    A high-level language such as Java, Groovy, Ruby or Python; we use Java and Groovy extensively \newline
    HTML5/CSS3/JavaScript, web standards, jQuery or frameworks like AngularJS would be great \newline
    HTML CSS and Javascript a must  \newline
    Experience with AJAX, XML, XSL, XSLT, CSS, JavaScript, JQuery, HTML and Web Services   \\ 
 \hline
\end{tabular}
\label{tab:skillinsent} % is used to refer this table in the text
\end{table}

We could see from the table, the technical close related concepts are always bing together. 
Based on such observation, we give a new statistical-based Ontology Similarity Measure. Two concepts $a$ and $b$ in the ontology,   their similarity $S_{a,b}$ could be the ratio of two factors:

\begin{enumerate}
    \item The ratio of the number of documents they exist together $N_{a \cap b}$ to the number of documents have a least one them $N_{a \cup b}$.
    \item The average $\log$ value of their minimum distance $mindis(doc,a,b)$ in documents that have them both. 
\end{enumerate}

$$ S(a,b) = \frac{  N_{a \cap b} / N_{a \cup b} }{avg(\log_2( mindis(doc,a,b) + 1 ))} $$

We only apply this measure on the concepts pair if: 
\begin{enumerate}
    \item The two concepts have the same direct 
    \item One concepts is the super  of another. 
\end{enumerate}
We set the restriction because the position of the concept in the ontology is defined based on their technical similarity to others. So similar technique will assigned into a same category.  If distance between two concepts are further than above situation we generally believe they are not related skills. The algorithm to calculate the similarity of two concepts is in algorithm ~\ref{alg:alg_similarity}.

\begin{algorithm}
\caption{Get Stat Similarity}
\label{alg:alg_similarity}
\KwIn{$Docs$�� $term1$, $term2$}
\KwOut{$similarity$}
$total=0$;
$hastwo=0$;
$dislist=\left [ ~~ \right ]$\;
\For{$i=1;~i~\le~len(Docs);~i++$}
{
  \If{ $ Docs_i~has~at~least~one~term $ }
    {
     $ total~+=~1 $ \;
     \If{ $ Docs_i~has~both~terms $ }
        {
           $ hastwo~+=~1 $ ;
           $ mindis~=~ minimium\_distance~(Docs_i, term1, term2) $ \;
           $ dislist.~add  ~\left(  log_2( mindis + 1 ) \right) $ \;
        }
    }
}
$ factor1~=~hastwo~ /~ total $  \;
$ factor2~=~ avg(dislist) $  \;
return $factor1 ~/~ factor2$\;
$ ~~ $
\end{algorithm}

