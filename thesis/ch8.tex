\chapter{CONCLUSION}
In this thesis�� we described JobFinder a personalized job-resume matching system, which could help job seeker to find appropriate jobs more easily. 
The key technical components of the system are information extraction and ontology matching.  

In the system, job descriptions and resumes will be processed by pipeline; and a finite automated tool will be used to extract the models from them. The models include fields like degree, major, skills. 
To find the appropriate jobs, similarities between the resume and job descriptions will be calculated.  The job descriptions will be matched the user's resume, the result will be sorted by the ontology similarity. Since the most appreciate jobs will be returned in front of other jobs, the users of the system could get better result than current job finding web site.


 The key technical
component that makes Prospect possible is the extraction
of various pieces of information from resumes. Since job
requirements often specify requirements such as at least 3
years of Java development" we use information extracted
from resumes to provide lters that allow screeners to nd
candidates that match such criteria. This allows us to overcome the limitations inherent in purely keyword based matching.
In piloting Prospect with screeners we estimated that using the tool roughly sped up the screening process by a factor
of 20 as compared to manual screening. This speedup can be
attributed mainly to a combination of two factors. Firstly,
ranking the candidates by match to the job description and
use of the lters provided based on various information extracted from the resume allows the screeners to inspect far
fewer resumes to shortlist a given number of candidates. Secondly, showing snippets of the resume based on its match to
the requirements of the job and based on the information
extracted from the resume allows a screener to shortlist or
reject candidates much faster than if they had to scan the
entire resume.
Despite these benets, we did receive qualitative feedback
that could help improving Prospect. One major area of improvement is in the consideration of three factors that the screeners considered to be important in the ranking function. In order of importance t
