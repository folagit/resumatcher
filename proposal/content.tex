


\section{MOTIVATION}
Currently the main channel for job seekers are online job finding web sites, like indeed�� monster etc. But most websites only allow users to use key word to search the jobs, which makes job searching as tedious and blind task. For example, I used keyword ��Java�� to search jobs in Mountain View, CA area on the job search engine indeed.com, the web site returned more than 7,000 jobs. Because the number of results of job searching is huge but un-ranked, the job seeker has to review every job description. Since no one has enough time to read all the jobs in the searching result, so the actual quality of job searching service is low. This is a classic problem of information overload.

My proposal is to create a web application which could use the resumes of job seekers to find the jobs that match their profiles best. The idea is to build a job candidate model, which should be generated from resumes and job descriptions. I want to transfer the job searching task from key word searching to candidate model matching. The matching result should be sorted by the matching score, higher matching score means a better matching. The job with higher matching score means the job is more appropriate to the job seeker, and if he applies for the job, the chance of getting the interview will be higher as well. The following is a figure of how this approach works.

\section{PREVIOUS WORK}

There several papers published on this topic, usually scholars call them Job Recommend System (JRS), because most of them used technology from other recommender systems. AlOtaibi et al. [1] summarized the categories of existing online recruiting platforms and listed the advantages and disadvantages of technical approaches in different JRSs.

They are some patents and systems related to this research, but most of them focusing on how to screen candidates. The systems are designed from the perspective of recruiter on how to find good candidates. Because the main revenue of current job finding websites is from the companies using their service, so their main interest is in how to promote more companies buying their services, not to improve the user experience of job seekers. By standing from the perspective of job seekers, this new approach can help find them the most suitable jobs.


\section{SYSTEM FEATURES}

The system should include such features:

\begin{enumerate}
    \item User could import his resume in different format, like txt, doc and pdf.
    \item User could search jobs by his resume. The search result will be sorted by the matching scores.
    \item User could set their searching preference, like location, company type, reputation, salary level, etc.
    \item The system should learn user��s preference during the user��s searching process.
    \item The system should be able to change the parameters of the matching algorithm adaptively according to the record of user's preference
    \item The user could set multiple searching agents, each one of which is a composition of different preferences, like one for remote location with high salary, another for local company with low salary.
\end{enumerate}


\section{METHODOLOGY}

I expect to use Natural Language Processing, especially semantic technique to parse the job description and resume, get information such as skill, specialties and background. This information will be used to build the model of job description and job seeker.  Ontology will be used to construct the knowledge base, which will include the taxonomy and rules, to support resume-job matching algorithm. Machine leaning algorithm will also be used to increase the accuracy of the matching algorithm by training with the existing data set.

The model of the candidate will include their specialties, working experience and education, all the information should be extracted from the resumes. The model also will be extracted from job description. When job seeker search the job, the system compares the candidate model from the resume to the models from the job descriptions.

In the initial phase I will only focus on the positions of IT job, because IT jobs have a special character,  skill set oriented, which means the person that the company want to hire must have some special skills and knowledge, like some programming languages, databases or software etc.
But simple keyword of skill name matching is far from enough, because job description and resume are both written in human language, even the same things, they could be written in different ways. For example, below is part of the resume of a job seeker, and a part of a job description:



\section{TIME TABLE}
\begin{center}
\begin{tabular}{ |c|p{1cm}|p{1cm}|p{1cm}|p{1cm}|p{1cm}|p{1cm}| }
 \hline
  Content                 & May              & Jun             & Jul              & Aug             & Sep             & Oct  \\ \hline
  Literature Review       & \cellcolor{red}  & \cellcolor{red} &                  &                 &                 &      \\ \hline
  Data Collection         &                  & \cellcolor{red} & \cellcolor{red}  &                 &                 &      \\ \hline
  Requirement Analysis    &                  &                 & \cellcolor{red}  &                 &                 &      \\ \hline
  Implementation          &                  &                 &                  & \cellcolor{red} &                 &      \\ \hline
  Evaluation and Analysis &                  &                 &                  &                 & \cellcolor{red} &   \\ \hline
  Writing and Wrap Up     &                  &                 &                  &      & \cellcolor{red} & \cellcolor{red}     \\ \hline
  Defending               &                  &                 &                  &                 &     &  \cellcolor{red}    \\ \hline


 \hline
\end{tabular}
\end{center}

 


\section{EVALUATION}

Two methods will be used to evaluate the quality of the system:
1.	Pre-collected data: The user will be asked to use current job hunting web sites to find some jobs that he believes to be the best match. These jobs will be put into the system with some other random collected jobs.  When searching the jobs with the resume of the same user, we can get precision, recall and F measure of the system. The ranking position of the result is also very important, so NDCG is also a very important measure.

2.	User experience:  The user will be asked to use both the system and current job finding web site. We can compare some factors to evaluate the system, such as:
 

\begin{enumerate}
    \item The time consumed to find satisfying jobs
    \item The quality of search results
    \item The user subjective experience with the both systems
\end{enumerate}

\section{CONCLUSION}

must have site here ~\cite{Hammond02}
